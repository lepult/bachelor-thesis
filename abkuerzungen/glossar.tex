\newglossaryentry{glossar}{name={Glossar},description={In einem Glossar werden Fachbegriffe und Fremdwörter mit ihren Erklärungen gesammelt}}
\newglossaryentry{glossaries}{name={Glossaries},description={Glossaries ist ein Paket was einen im Rahmen von LaTeX bei der Erstellung eines Glossar unterstützt}}
\newglossaryentry{HTTP}{name={HTTP},description={Das Hypertext Transfer Protocol ist ein grundlegendes Protokoll für den Datenaustausch im World Wide Web, das die Übertragung von Hypertext-Dokumenten zwischen Webbrowsern und -servern ermöglicht und durch Zustandslosigkeit, verschiedene Methoden (GET, POST), Statuscodes und die Verwendung von URLs gekennzeichnet ist}}
\newglossaryentry{MQTT}{name={MQTT},description={Message Queuing Telemetry Transport ist ein leichtgewichtiges Netzwerkprotokoll für die effiziente Nachrichtenübertragung zwischen Geräten, besonders in Machine-to-Machine (M2M) und Internet of Things (IoT)-Anwendungen}}
\newglossaryentry{Webhook}{name={Webhook},description={Ein Webhook ist eine automatisierte Methode zur Übertragung von Echtzeitinformationen zwischen Webdiensten durch das Senden von HTTP-POST-Anfragen an vordefinierte URLs, um sofortige Ereignisbenachrichtigungen zu ermöglichen}}
\newglossaryentry{Websocket}{name={Websocket}, description={Ein Kommunikationsprotokoll, das eine bidirektionale, echtzeitfähige Verbindung zwischen einem Webbrowser und einem Server ermöglicht, wodurch eine kontinuierliche Datenübertragung für interaktive Webanwendungen in Echtzeit ermöglicht wird}}
\newglossaryentry{Mockup}{name={Mockup}, description={Eine visuelle Darstellung oder Modellierung eines Designs, Produkts oder einer Benutzeroberfläche, die dazu dient, das Aussehen und die Funktionalität zu skizzieren und Feedback zu sammeln, bevor die eigentliche Entwicklung beginnt}}
\newglossaryentry{Base64}{name={Base64}, description={Base64 ist ein Kodierungsverfahren, das Binärdaten in eine Textdarstellung umwandelt, indem 64 verschiedene ASCII-Zeichen verwendet werden, wodurch die Daten für die Übertragung über textbasierte Protokolle geeignet gemacht werden.}}