\newglossaryentry{HTTP}{name={HTTP},description={Das Hypertext Transfer Protocol ist ein grundlegendes Protokoll für den Datenaustausch im World Wide Web, das die Übertragung von Hypertext-Dokumenten zwischen Webbrowsern und -servern ermöglicht. Es ist durch Zustandslosigkeit, verschiedene Methoden – wie GET und POST –, Statuscodes und die Verwendung von URLs gekennzeichnet}}
\newglossaryentry{MQTT}{name={MQTT},description={Message Queuing Telemetry Transport ist ein leichtgewichtiges Netzwerkprotokoll für die effiziente Nachrichtenübertragung zwischen Geräten, besonders in Machine-to-Machine und Internet of Things Anwendungen}}
\newglossaryentry{Webhook}{name={Webhook},description={Ein Webhook ist eine automatisierte Methode zur Übertragung von Echtzeitinformationen zwischen Webdiensten durch das Senden von HTTP-Anfragen an vordefinierte URLs. Webhooks ermöglichen sofortige Ereignisbenachrichtigungen}}
\newglossaryentry{Websocket}{name={Websocket}, description={Websocket ist ein Kommunikationsprotokoll, das eine bidirektionale, echtzeitfähige Verbindung zwischen einem Webbrowser und Server ermöglicht}}
\newglossaryentry{Mockup}{name={Mockup}, description={Ein Mockup ist eine visuelle Darstellung oder Modellierung eines Designs, Produkts oder einer Benutzeroberfläche, die dazu dient, das Aussehen und die Funktionalität zu skizzieren und Feedback zu sammeln, bevor die eigentliche Entwicklung beginnt}}
\newglossaryentry{Base64}{name={Base64}, description={Base64 ist ein Kodierungsverfahren, das Binärdaten in lesbare ASCII-Zeichen umwandelt und dabei 64 verschiedene Zeichen verwendet, um die Übertragung von Daten über textbasierte Protokolle zu ermöglichen}}
\newglossaryentry{DOM}{name={DOM}, description={Das Document Object Model ist eine Programmierschnittstelle, die die Struktur und Inhalte eines HTML- oder XML-Dokuments repräsentiert und es ermöglicht, auf diese Elemente zuzugreifen, sie zu ändern und zu manipulieren}}
\newglossaryentry{Mixins}{name={Mixins}, description={Sass Mixins sind wiederverwendbare Code-Ausschnitte in Sass, die dazu dienen, CSS-Eigenschaften und -Werte zu kapseln und sie leicht in verschiedenen Stilen oder Elementen anzuwenden}}
\newglossaryentry{MQTT-Broker}{name={MQTT-Broker}, description={Ein MQTT-Broker ist eine Middleware-Komponente, die als Vermittler fungiert, um Nachrichten zwischen verschiedenen Clients über das MQTT-Protokoll auszutauschen und zu verwalten}}
\newglossaryentry{Microservice}{name={Microservice}, description={Ein Microservice ist eine isolierte und eigenständige Softwarekomponente, die eine spezifische Aufgabe erfüllt und innerhalb einer verteilten Architektur betrieben wird}}
\newglossaryentry{MQTT-Topic}{name={MQTT-Topic}, description={Ein MQTT-Topic ist eine eindeutige Zeichenkette, die verwendet wird, um Nachrichten innerhalb des MQTT-Protokolls zu adressieren und zu organisieren. Clients nutzen MQTT-Topics, um Nachrichten auszutauschen}}