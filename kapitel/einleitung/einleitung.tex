\section{Einleitung}
Der Einsatz von Robotern prägt zunehmend den Arbeitsmarkt und beeinflusst die Art und Weise wie Unternehemn ihre Prozesse gestalten. Diese Entwicklung beschränkt sich nicht nur auf die Industrie und den Verbrauchermarkt, sondern immer zunehmender auch auf die Dienstleistungsbranche. So ist der Absatz an Service Robotern für professionelle Anwendungen 2022 um 48\% gestiegen(Quelle: IFR). Hierbei gibt es eine Abgrenzung zu Service Roboter für Verbraucher, bei denen der Absatz 2022 gesunken ist und Industrieroboter bei denen der Absatz 2022 gestiegen ist. Aufgrund der Stichprobenentnahme für die Berechnung des Absatzes der Service Roboter für professionelle Anwendungen ist keine genauere Vergleichbarkeit mit den restlichen Absatzdaten von Robotern möglich.(Quelle: IFR) Das Wachstum des Absatzes scheint unter anderem durch einen Mangel an Arbeitskräften getrieben. So ist die Menge unbesetzter Abreitsplätze in den letzten Jahren stark gestiegen(Quelle: ...). Service Roboter bieten eine hierbei effiziente Unterstützung der Arbeiter, ersetzen diese aber nicht vollständig. Bodenreinigungsroboter werden Beispielsweise in Kombination mit Menschen eingesetzt, sodass der Roboter die großen Flächen und der Mensch die Kanten reinigt.(Quelle: IFR)
% TODO Add sources

\subsection{Hintergrund und Motivation}
Fehlender-Text

\subsection{Zielsetzung und Forschungsfrage}
Fehlender-Text

\subsection{Methodik}
Fehlender-Text
