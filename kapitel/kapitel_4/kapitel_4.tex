\newpage
\section{Implementierung}
Im Folgenden wird die Implementierung, sowie die Evaluierung des Prototyps beschrieben.

Fehlender-Text

\subsection{Implementierung des Prototyps}
% ggf. wird ein weiteres Prototyp-Backend erstellt in dem die Daten abgespeichert werden, die aktuell hard-coded sind. Falls dieses Backend erstellt wird, sollte das hier noch erwähnt werden
Der Prototyp besteht aus dem Frontend, dass auf das \ac{BCB} zugreift, um die Roboterdaten anzufragen. Wie dieses mit den Robotern kommuniziert, wurde in einem vorherigen Kapitel erklärt. Bei dem Frontend handelt es sich um eine Webanwendung, die auf AWS gehostet ist und auf chayns-Seiten eingebunden werden kann. Für die Entwicklung wurde das Web-Framework React JS genutzt, welches auf HTML, CSS und Javascript basiert. Statt Javascript wurde Typescript eingesetzt, da es durch statische Typisierung eine frühzeitige Fehlererkennung und eine bessere Wartbarkeit des Codes bietet. Für eine bessere Übersichtlichkeit sind die Stylesheets in SCSS statt in CSS geschrieben. Die Zustandsinformationen der Anwendung werden zentral mithilfe von React Redux gespeichert. Die Grundstruktur des Frontend-Projekts wurde mihilfe des Shell Befehls "npx create-chayns-app" generiert. Mithilfe des npm-Pakets "chayns-toolkit" wurde die Kompilierung des Codes konfiguriert. Die eingesetzten GUI-Elemente wie Buttons und Aufklapper werden durch die Komponentenbibliothek "chayns-components" bereitgestellt. Die Typescript Typen der Objekte, die aus dem \ac{BCB} empfangen werden, konnten aus einem anderen Projekt kopiert werden.

\subsubsection{Verwaltungsansicht}
In der Verwaltungsansicht gibt es eine Übersicht über die existierenden Roboter und Stockwerke. So gibt es zum einen eine Liste der existierenden Roboter, über die man Zugriff auf verschiedene Informationen hat. So sind hier unter anderem auch Informationen ersichtlich, auf die man wegen fehlender Berechtigungen in den anderen Anischten keinen Zugriff hat. Hierbei handelt es sich um Informationen wie die Systeminterne \ac{ID} und die Neustart-Zeit des Roboters. Über diese Liste lassen sich zusätzlich auch Einstellungen der Roboter ändern. So hat der Administrator hier beispielsweise die Möglichkeite den Namen, den Ausgabepunkt oder die Homebase des Roboters zu ändern. Auch gibt es in dieser Ansicht eine mit \deckgl implementierte Vorschau der Karte, in der die Position des Roboters, sowie die Positionen der Roboterdaten ersichtlich sind. Da sich die Roboter in dieser Ansicht nicht steuern lassen, gibt es außerdem einen Button, über den zur Kellneransicht navigiert wird.

Bis auf die Vorschau der Karte ist die Verwaltungsansicht nicht besonders komplex. Da die Daten der Roboter und Stockwerke sauber strukturiert sind, lassen sich die Informationen mithilfe von React leicht mappen. Für die Vorschau der Karte konnte die Kellneransicht wiederverwendet werden. Ein paar Features wie das Steuern der Roboter ist hierbei einfach nur deaktiviert.

\paragraph{Import der 3D-Modelle}

\subsection{Synchronisierung des Gebäudemodells und der Roboterdaten}
Im vorherigen Kapitel wurde erläutert, warum die Synchronisierung der Roboterdaten mit den 3D-Modellen manuell durchgeführt werden muss. Es wurde ein Editor implementiert und im Prototyp eingebunden, mit dem der Nutzer die 3D-Modelle in Relation zu den Roboterdaten positionieren und rotieren kann.

Der Editor nutzt, so wie andere Teile der Anwendung \deckgl um 3D-Modelle und Roboterdaten dazustellen. In \deckgl bildet die x-Koordinate die Ost-West-Achse, die y-Koordinate die Nord-Süd-Achse und die z-Koordinate die Höhe ab. Mithilfe der in \deckgl integrierten Events onDragStart, onDrag und onDragEnd kann das angeklickte Objekt per Ziehen der Maus verschoben und rotiert werden. Beim Drücken der Steuerungstaste wird das angeklickte Modell verschoben und beim Drücken der Shift-Taste rotiert. Beim Drücken der Steuerungs- oder Shift-Taste werden die Objekte durchsichtigt um zu kommunizieren, dass diese nach dem anklicken bewegt werden können. Das Verschieben und Rotieren kann mithilfe der Tastenkombination Strg + Z rückgängig gemacht und mit Strg + Y wiederholt werden. Hierfür existieren zwei Stapel in denen vergangene Aktionen gespeichert werden.
% Initiale Ansicht Position kann eingestellt werden

Die 3D-Modelle können an der x- und y-Achse verschoben und an der y-Achse rotiert werde. Warum keine anderen Transformationen nötig sind, wird im folgenden Abschnitt erläutert.

\subsubsection{Unnötige Transformationen}

Der Boden der 3D-Modelle liegt relativ konstant auf der z-Koordinate 0. Aufgrund der Ungenauigkeiten die durch den \ac{LiDAR}-Scan entstehen ist der Boden aber nicht vollständig eben. Da die Roboterdaten zweidimensional und somit keinen Höhenwert haben besteht die Gefahr, dass diese an manchen Stellen unter den 3D-Modellen verschwinden, wenn auf der z-Koordinate 0 positioniert werden. Aus diesem Grund werden die Roboterdaten stattdessen an der z-Koordinate 0,1 positioniert. Da der Boden bei allen 3D-Modellen an derselben z-Koordinate positioniert ist, müssen diese durch den Nutzer nicht weiter an der z-Achse verschoben werden.

Im Vergleich zu den Roboterdaten sind die 3D-Modelle immer um 90° an der z-Achse und einen beliebigen Wert an der y-Achse rotiert. Die Rotation an der x-Achse ist bereits synchron. Die 3D-Modelle werden automatisch um -90° an der z-Achse rotiert und müssen vom Nutzer somit nur nach an der y-Achse rotiert werden.

Da sich die Roboterdaten und 3D-Modelle denselben Maßstab teilen müssen diese durch den Nutzer nicht skaliert werden.

\newpage
\subsection{Evaluierung des Prototyps}
Fehlender-Text