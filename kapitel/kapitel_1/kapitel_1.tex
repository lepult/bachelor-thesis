\newpage
\section{Grundlagen} \label{Grundlagen}
Fehlender-Text

\subsection{Service Roboter}
In den letzten Jahren haben Service Roboter in verschiedenen Branchen an Bedeutung gewonnen. Ein Grund hierfür ist der technische Fortschritt der Robotik in kombination mit KI, Big Data, Kameras, Sensoren und Spracherkennung \cite[S.~424]{Paluch2020}. Dieser Abschnitt gibt einen Überblick über die Grundlagen von Service Robotern und eine Einordnung in die Robotik. Hierbei wird ein besonderer Fokus auf den Einsatz in der Gastronomie gesetzt.

\subsubsection{Definition}
In wissenschaftlichen Arbeiten werden viele verschiedene Definitionen für Service Roboter genutzt. In dieser Arbeit wird mit der Definition aus der ISO Norm 8373:2021 \cite[Kap.~3]{ISO2021} gearbeitet. Nach dieser handelt es sich bei Service Roboter um Roboter die im privaten oder professionellen Gebrauch nützliche Aufgaben für Menschen oder Equipment erledigen. Hierbei werden Service Roboter außerdem von Industrierobotern und Medizinrobotern abgegrenzt. Die \ac{IFR} \cite{IFR2024} ergänzt die Voraussetzung, dass Service Roboter voll- oder zumindest teilautonom handeln können. Unter dem professionellen Einsatz von Service Robotern versteht man solche, die kommerziell eingesetzt werden \cite[S.~4]{GonzalezAguirre2021}, beispielsweise in den Bereichen Gesundheitswesen, Landwirtschaft oder Tourismus \cite[S.~9]{GonzalezAguirre2021}.

\subsubsection{Einsatzmöglichkeiten}
Service Roboter werden bereits in vielen Bereichen eingesetzt. So gibt es verschiedene Beispiele in denen Service Roboter in Hotels für den Gästeempfang, Check-in und Gepäcklieferung eingesetzt werden. Auch werden Sie an Flughäfen für die Beratung von Reisenden, Scannen von Boardingpässen, Check-in, Bodenreinigung und Patrouilliengänge genutzt. In der Pflege helfen Service Roboter den Pflegern beim Heben von Patienten. Auch können sie Übungen mit Patientengruppen durchführen und angenehme Gespräche starten \cite[S.~425-427]{Paluch2020}. Aufgaben mit geringer kognitiver und emotionaler Komplexität können Service Roboter hierbei vollautonom und ohne Aufsicht durch einen Menschen durchführen\cite[S.~429]{Paluch2020}. Hierbei handelt es sich beispielsweise um Aufgaben wie Staubsaugen, Rasenmähen oder Gepäcklieferung. Auch bestimmte Aufgaben, die Service Roboter in der Gastronomie übernehmen, können zumindest teilautonom von Service Robotern ausgeführt werden. Wie bereits erwähnt beschäftigt sich diese Arbeit mit Robotern von Pudu. Pudu stellt Service Roboter her, die in der Gastronomie zum Begrüßen und Geleiten von Gästen, zum Liefern bestellter Speisen und Getränke, zum Zurückbringen dreckigen Geschirs und zum Putzen des Bodens eingesetzt werden können \cite{PUDU2024}. Während die Roboter zum Putzen des Bodens autonom eingesetzt werden können, braucht es beim Liefern und Zurückbringen Menschen, die die Gegenstände dem Roboter geben und abnehmen. Diese Roboter arbeiten somit teilautonom.

\subsubsection{Pudu Roboter Funktionen}

\subsubsection{Bot Control Backend}
Im Rahmen dieser Arbeit wird eine Schnittstelle zwischen den Pudu Robotern und der prototypischen Webanwendung genutzt. Diese Schnittstelle ist das sogenannte \ac{BCB}. In diesem Abschnitt wird die Verbindung zwischen dem \ac{BCB} und den Pudu Robotern erläutert. Im weiteren Verlauf dieser Arbeit wird nicht auf diese Verbindung eingegangen, sondern stattdessen nur auf die einzelnen Endpunkte des \ac{BCB}, sobald diese genutzt werden.

% TODO Schaubild hinzufügen
Die folgenden Informationen zum Service Framework von Pudu zwischen Robotern und \ac{BCB} stammen aus dem SDK Guidance Document von Pudu \cite{PuduSDK}. Dieses steht nicht im Internet zur Verfügung. Das \ac{BCB} nutzt das Service Framework von Pudu, um mit den Robotern zu kommunizieren. Hierbei hat das \ac{BCB} nur eine direkte Verbindung zum Node.js Microservice. Roboterdaten fragt das \ac{BCB} über eine \gls{HTTP}-Anfrage an den Microservice an. Der Node.js Microservice leitet diese Anfrage via \gls{MQTT} an die PUDU Cloud weiter. Diese beantwortet die Anfrage mit den angefragten Daten. Manche Anfragen fordern Daten direkt von den Robotern an. In diesem Fall leitet die PUDU Cloud diese Anfragen per \gls{MQTT} an die Roboter weiter, die diese dann beantworten. Die Roboter können Ereignisse auch unaufgefordert an das \ac{BCB} verschicken. Hierfür muss das \ac{BCB} die Adresse, an welche die Daten geschickt werden sollen, im Microservice als \gls{Webhook} registrieren. Hierbei muss auch angegeben welcher Ereignis-Typ von welchem Roboter an den Endpunkt gesendet werden soll.

Das \ac{BCB} dient nicht nur als Schnittstelle zu den Pudu Robotern, sondern abstraiert neue Funktionen aus den Funktionen die Pudu bietet. Zum einen gibt es die Möglichkeit einen Lieferroboter direkt zu einem Lieferpunkt in einem anderen Stockwerk zu schicken. Zum anderen kann der Roboter vor einer geschlossenen halten, diese öffnen und dann weiter fahren. Diese beiden Funktionen bietet Pudu nicht.

\newpage
\subsection{Webanwendungen}
Fehlender-Text


\newpage
\subsection{3D Modelle}
Fehlender-Text

\subsubsection{Arten von 3D Modellen}
Fehlender-Text
% Punktwolke und Polygonnetz erklären
% Erwähnen, dass bestimmte Scann/Photogrammetrie (?) Methoden Punktwolken ausgeben
% Diese eignen sich allerdings nicht für die Einbindung zusätzlicher Daten
% Daher Polygonnetze am besten geeignet
% Methoden die Punktwolken erzeugen automatisch ungeeignet

\subsubsection{Generierung}
Fehlender-Text

\paragraph{Fotogrammetrie}
Fehlender-Text

\paragraph{LiDAR Scanning}
Fehlender-Text

\paragraph{Scan2Scene}
Fehlender-Text

\paragraph{Modellierungs-Software}
Fehlender-Text


\subsubsection{Einbindung im Web}
Fehlender-Text

\paragraph{WebGL}
Fehlender-Text

\paragraph{Three.js}
Fehlender-Text

\paragraph{deck.gl}
Fehlender-Text