\newpage
\section{Technische Herausforderungen}
Vor und während der Implementierung des Prototyps sind verschiedene größere technische Herausforderungen aufgetreten, die gelöst werden mussten. Die Herausforderungen werden in diesem Kapitel mit den gewählten Lösungsansätzen vorgestellt. 

\subsection{Methode zur Modellgenerierung}
Die 3D-Modelle der Stockwerke wurden mithilfe des LiDAR-Scannens erstellt. Hierfür wurde die Scaniverse App auf einem iPhone 14 Pro verwendet. Da das Gerät zur Verfügung stand und die Scaniverse App kostenlos nutzbar ist, war das Scannen mit keinen Kosten verbunden. Aufgrund des höheren Aufwands bei der Aufnahme und Verarbeitung von Bildern wurde die Fotogrammetrie als Methode verworfen. Zwar würde sich das iPhone auch für Aufnahmen eignen die fotogrammetrisch verarbeitet werden können, diese Verarbeitung erfordert allerdings ein weiteres Gerät mit einer leistungsstarken \ac{GPU}. Außerdem ist das Verarbeiten der Bilder bei der Fotogrammetrie mit mehr Aufwand verbunden als beim \ac{LiDAR}-Scannen. Die \ac{KI}-gestützten Methoden wurden nicht eingesetzt, da sie aufgrund der geringen Menge an Publikationen noch als unausgereift erscheinen.

\subsection{3D Visualisierung im Web}
Für das Einbinden von 3D-Visualisierungen im Web gibt es verschiedene Ansätze und Technologien. In diesem Abschnitt wird die Auswahl von \deckgl{} als Visualisierungs-Framework, sowie die Wahl des Dateiformats der 3D-Modelle erläutert.

\subsubsection{deck.gl}
Für die Umsetzung der 3D-Visualisierung im Prototyp wurde das Framework \deckgl{} gewählt. Es wurde gewählt, da die Anwendung als Karte genutzt werden soll und \deckgl{} für das Entwickeln dieser ausgelegt ist. So ist die Navigation und das Verhalten der Kamera bereits passend konfiguriert, sodass bei der Entwicklung dieser Features Zeit gespart werden kann. Vor der Implementierung des Prototyps konnte bestätigt werden, dass die Anforderungen an den Prototyp mit dem Framework eingehalten werden können.

\subsubsection{Dateiformat der 3D-Modelle}\label{sec:ModelFileFormat}
Für die Darstellung eines 3D-Modells gibt es in \deckgl{} zwei Möglichkeiten: das Einbinden des \ac{OBJ} Dateiformats in der SimpleMeshLayer \cite{DeckglSimpleMeshLayer} und das Einbinden des \ac{glTF} Dateiformats in der ScenegraphLayer \cite{DeckglScenegraphLayer}. Beide Dateiformate werden in der Scaniverse App für den Dateiexport angeboten.

Das \ac{OBJ} Format besteht aus einer Datei mit der Endung \obj{}, in der die dreidimensionalen geometrischen Formen kodiert sind \cite{OBJSpec} und einer Datei mit der Endung \mtl{}, in der die optischen Materialeigenschaften und Texturierung kodiert sind \cite{MTLSpec}. Für das Einbinden des \ac{OBJ} Dateiformats wird die \loadersgl{} Programmbibliothek benötigt, die allerdings nur die \obj{} Datei und nicht die \mtl{} Datei parsen kann \cite{OBJLoader}. So können die 3D-Modelle in der SimpleMeshLayer nur ohne Textur angezeigt werden.

Das \ac{glTF} Format bietet zwei verschiedene Dateiformate, wobei hier nur die binäre Variante relevant ist. Diese besteht aus einer Datei mit der Endung \glb{}, welche neben den geometrischen Formen auch die Materialeigenschaften und Texturierung enthält. Das \ac{glTF} Format verspricht eine geringere Dateigröße als vergleichbare Dateiformate wie \ac{OBJ}.\cite[Abschnitt 2]{glTFSpec} Mithilfe der \loadersgl{} Programmbibliothek lassen sich 3D-Modelle des Formats ohne großen Aufwand in der ScenegraphLayer von \deckgl{} einbinden \cite{DeckglScenegraphLayer}. Da ein 3D-Modell mit passender Texturierung eine bessere Übersichtlichkeit bietet und \ac{glTF} Dateien eine geringere Dateigröße haben, wird die ScenegraphLayer mit 3D-Modellen im \ac{glTF} Format für die Darstellung der Raummodelle genutzt.

Die \ac{glTF} Dateien, die für den Prototyp aus der Scaniverse App exportiert werden, sind mit einer Dateigröße von 14 bis 21 \ac{mB} für den Einsatz im Prototyp zu groß, da sie – vermutlich durch ihre Komplexität – nicht auf Mobilgeräten angezeigt werden können. Außerdem beeinflusst die Dateigröße die Ladezeiten negativ – sowohl beim Herunterladen der Daten vom Webserver als auch beim Anzeigen der 3D-Modelle. Aus diesem Grund müssen die Dateien komprimiert werden. Hierfür sind die 3D-Modelle, die im Prototyp eingesetzt werden mit dem OptimizeGLB Online Konverter \cite{OptimizeGLB} manuell komprimiert worden. Insbesondere durch den Einsatz des \ac{WebP} Bildformats für Texturen werden die Dateien effektiv komprimiert. Die komprimierten Dateien sind zwischen 330 und 550 \ac{kB} groß, was einer Kompression von über 95\% entspricht. Die Qualität der komprimierten 3D-Modelle ist erkennbar geringer, reicht aber für den Zweck der Anwendung trotzdem aus, da größere Merkmale weiter gut erkennbar sind und somit die Übersichtlichkeit weiter garantiert wird.

\subsection{Synchronisierung des Gebäudemodells und der Roboterdaten}
Im Prototyp sollen die Roboterdaten in den 3D-Modellen integriert dargestellt werden. Die Roboterdaten und 3D-Modelle haben den gleichen Maßstab, die Positionen und Rotationen der Datensätze stimmen allerdings nicht miteinander überein. Diese Unterschiede sind eine Konsequenz daraus, dass zur Generierung unterschiedliche Scanning-Methoden eingesetzt werden. Außerdem stimmen die Positionen der 3D-Modelle nicht untereinander überein, da beim Scannen an verschiedenen Ausganspunkten angefangen wird.

Aus diesem Grund müssen die Positionen der Roboterdaten mit den 3D-Modellen synchronisiert werden. Auch müssen die Positionen der 3D-Modelle untereinander synchronisiert werden. Eine automatische Synchronisierung ist aus verschiedene Gründen zu komplex. Zum einen sind die Formate der Daten zu verschieden, denn während die 3D-Modelle aus komplexen dreidimensionalen Formen bestehen, setzen sich die Roboterdaten aus zweidimensionalen Linien und Punkten zusammen. Zum anderen gibt es in beiden Datensätzen unterschiedliche Ungenauigkeiten in Bezug auf die Realität. Sowohl \ac{VSLAM}, mit dem die Roboterdaten untereinander positioniert werden, als auch das \ac{LiDAR}-Scanning, mit dem die 3D-Modelle generiert werden, sind fehlerbehaftet. Da sich diese Scanning-Methoden unterscheiden, unterscheiden sich auch diese Ungenauigkeiten.

Da eine automatische Synchronisierung der Datensätze somit ausgeschlossen ist, muss diese manuell durch den Nutzer vorgenommen werden. Deshalb wurde ein Editiermodus implementiert, mit dem der Administrator die 3D-Modelle und Roboterdaten durch Verschieben und Rotieren der 3D-Modelle synchronisieren kann. Die Implementierung wird im Abschnitt \ref{sec:EditMode} beschrieben.
