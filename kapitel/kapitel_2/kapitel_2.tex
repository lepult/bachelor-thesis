\newpage
\section{Anforderungen an den Prototyp}\label{sec:Requirements}
In diesem Kapitel werden die funktionalen und nicht-funktionalen Anforderungen an den Prototyp, sowie die Anforderungen an die Modellerzeugungsmethode vorgestellt.

\subsection{Anforderungen an Modellerzeugung}

Für die Relevanz des Prototyps ist es erheblich, dass es eine einfache Methode gibt, mit der 3D-Gebäudemodelle erzeugt werden können. Die Methode soll möglichst geringe Kosten verursachen und kein besonderes technisches Know-how erfordern, um sich von klassischer Modellierungssoftware wie Blender abzuheben. Die erzeugten Modelle müssen nicht hundertprozentig genau, aber genau genug sein, um eine gute Übersicht zu ermöglichen. Es soll eine Methode gefunden werden, die diese Anforderung erfüllt. Eine tiefere Analyse der gewählten Methode sowie ein tieferer Vergleich zwischen verschiedenen Methoden ist unerheblich, da die gewählte Methode nicht die beste sein muss. So soll nur herausgearbeitet werden, dass es eine passende Methode gibt, nicht welche am besten für den Anwendungsfall geeignet ist.

\subsection{Funktionale Anforderungen}\label{sec:FunctionalRequirements}
Wie im Abschnitt \ref{sec:ResearchQuestion} beschrieben, soll der Prototyp als Webanwendung implementiert werden. So wird mit wenig Aufwand eine Plattformunabhängigkeit ermöglicht. Diese ist nötig, da die Anwendung auf beliebigen Geräten nutzbar sein soll. Aus diesem Grund soll der Prototyp außerdem responsiv sein.

Die Anwendung lässt sich in drei zentrale Funktionen aufteilen: Übersicht, Steuerung und Verwaltung.

\subsubsection{Übersicht}

In der Übersicht sollen die relevanten Roboterdaten zusammen mit dem Gebäudemodell in einer dreidimensionalen Ansicht abgebildet werden. Da die Roboter in der Lage sind Fahrstuhl zu fahren, soll es die Möglichkeit geben zwischen verschiedenen Stockwerken zu navigieren. Es soll außerdem in Echtzeit angezeigt werden, wo sich die Roboter befinden. In der dreidimensionalen Darstellung sollen die Roboter dann so wie in der Realität fahren. Auch soll ersichtlich sein, ob und womit die Roboter beschäftigt sind. Falls ein Roboter einen Lieferauftrag ausführt, soll das Ziel angezeigt werden. Weitere Daten, die von den Robotern stammen und angezeigt werden sollen, sind die Positionen aller möglichen Ausgabe- und Lieferpunkten, sowie Ladestationen. Zudem haben die Roboter festgelegte Pfade, an denen sie sich beim Fahren orientieren. Diese sollen auch angezeigt werden.

\subsubsection{Steuerung}

Mit der Steuerung sollen die Roboter beauftragt werden können. So soll es die Möglichkeit geben einen Zielpunkt einzustellen. Auch sollen die Roboter an ihre Ladestation geschickt werden können.

\subsubsection{Verwaltung}

In der Verwaltung sollen sowohl die Roboter als auch die Übersicht selbst verwaltet werden können. So soll das Ändern der verschiedenen Roboter-Einstellungen möglich sein. Beispielsweise soll die eingestellte Ladestation änderbar sein. Es soll außerdem möglich sein die 3D-Gebäudemodelle zu importieren und diese mit den Roboterdaten zu synchronisieren. So sollte es die Möglichkeit geben das importierte 3D-Gebäudemodell und die mit \ac{VSLAM} erzeugte interne Karte der Roboter anzugleichen, damit die Positionen der Roboter in der Übersicht mit der Realität übereinstimmen.

\subsubsection{Benutzer}

Für die Anwendung gibt es drei verschiedene Nutzergruppen: Gäste, Mitarbeiter und Administratoren. Der für den Nutzer verfügbare Funktionsumfang erhöht sich in dieser Reihenfolge. So sollen Gäste nur einen eingeschränkten Zugriff auf die Übersicht haben während Administratoren Zugriff auf alle Funktionen der Übersicht, Steuerung und Verwaltung haben.

Der Wert von Servicerobotern muss insbesondere aus der Sicht von Gästen noch bewiesen werden \cite[S.~429]{Paluch2020}. Deshalb sollen Gäste einen eingeschränkten Zugriff auf die Übersicht der Roboter bekommen, in der sie die Positionen und aktuellen Aufträge sehen können. Mithilfe dieser Transparenz sollen ihnen die Vorteile von Servicerobotern veranschaulicht werden.

Die Mitarbeiter sollen einen vollständigen Zugriff auf die Übersicht und Steuerung bekommen. So werden ihnen alle Funktionen zur Verfügung gestellt, die sie für das Steuern der Roboter brauchen. Auch sollen ihnen in der Übersicht mehr Informationen angezeigt werden. Einen Zugriff auf die Verwaltung der Roboter brauchen die Mitarbeiter nicht.

Auf die Verwaltung sollen nur Administratoren Zugriff haben. Während die Verwaltung der Roboter kontinuierlich genutzt werden sollte, sollte die Verwaltung der Stockwerke hauptsächlich bei der Einrichtung gebraucht werden. So sollte die Karte nur zu Beginn eingerichtet und danach nie wieder verändert werden müssen.

\subsection{Nicht-funktionale Anforderungen}
Lange Ladezeiten während der Nutzung der Anwendung verursachen Frust. Die Reduktion der Ladezeiten ist deshalb eine zentrale Anforderung an den Prototyp. Normalerweise lassen sich Ladezeiten sowohl auf der Server- als auch auf der Client-Seite verbessern. Der Schwerpunkt des Prototyps liegt allerdings im Frontend, weshalb sich Optimierungen nur auf die Client-Seite beschränken sollen. Das \ac{BCB} soll unabhängig davon, ob Optimierungspotenzial existiert, nicht verändert werden. Maßnahmen, die sich so anbieten sind das Verzögern des Ladens nicht essenzieller Ressourcen, sowie die Reduktion des Datenverbrauchs. Eine Reduktion des Datenverbrauchs bietet auch den Vorteil, dass sich Kosten für Nutzer reduzieren, die einen teuren oder begrenzten Datenplan verwenden. Unabhängig von Ladezeiten sollte der Prototyp trotz der rechenaufwändigen Darstellung der 3D-Modelle möglichst performant sein. So soll es beispielsweise beim Navigieren keine Ruckler geben. All diese Anforderungen lassen sich unter dem Begriff der Effizienz zusammenfassen. Die Effizienz soll anhand der wahrgenommenen Ladezeit, der Lade-Reaktionsfähigkeit und der Smoothness bewertet werden. Neben der Effizienz sollte die Anwendung auch benutzerfreundlich sein, damit sichergestellt wird, dass sich die Benutzer problemlos zurechtfinden können. Die Benutzerfreundlichkeit soll anhand von Usability Tests bewertet werden.
