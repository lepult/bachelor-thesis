\newpage
\section{Anforderungen an den Prototyp}
In diesem Kapitel werden die Anforderungen an den Prototyp vorgestellt. Die Anforderungen wurden basierend auf der Wissensbasis aus der Literaturrecherche erstellt.

\subsection{Abgrenzungen und Einschränkungen}

Für die Relevanz des Prototyps ist es erheblich, dass es eine einfache Methode gibt, mit der 3D-Gebäudemodelle erzeugt werden können. Hierfür soll eine Methode gefunden werden, die diese Anforderung erfüllt. Eine tiefere Analyse der gewählten Methode ist hierbei nicht nötig. Auch ist ein tieferer Vergleich zwischen verschiedenen Methoden zur Auswahl der besten Methode unerheblich, da die gewählte Methode nicht die beste sein muss. In dieser Arbeit soll nur herausgearbeitet werden, dass es eine passende Methode gibt, nicht welche am besten für den Anwendungsfall geeignet ist.

\subsection{Funktionale Anforderungen}

Wie bereits in der Einleitung beschrieben, soll der Prototyp als Webanwendung implementiert werden. So wird eine Plattformunabhängigkeit mit geringem Aufwand garantiert. Diese wird gebraucht, damit Kellner die Anwendung auf beliebigen Smartphones nutzen können. Auch an Desktop Computern soll die Anwendung insbesondere zur Verwaltung der Roboter genutzt werden können. Entsprechend soll die Webanwendung responsiv sein, um auf Gerätetypen mit beliebigen Bildschirmgrößen zu funktionieren.

Die Anwendung lässt sich in drei zentrale Funktionen aufteilen: Übersicht, Steuerung und Verwaltung.

\subsubsection{Anforderungen an Modellerzeugung}
Fehlender Text
% Billig
% Wenig Aufwand
% Kein technisches Know How nötig
% erzeugt relativ genaue Modelle => Modelle müssen nicht 100% genau sein, aber gut genug sein, dass die Modelle und Roboterdaten zusammenpassen

\subsubsection{Übersicht}

In der Übersicht sollen die - für die entsprechende Nutzergruppen - relevanten Roboterdaten in einer 3-dimensionalen Ansicht zusammen mit dem Gebäudemodell abgebildet werden. Da die Roboter in der Lage sind Fahrstuhl zu fahren soll es die Möglichkeit geben zwischen verschiedenen Stockwerken zu navigieren. Es soll außerdem in Echtzeit angezeigt werden, wo sich die Roboter befinden. In der Darstellung sollen die Roboter dann so wie in der Realität fahren. Auch soll ersichtlich sein, ob und womit die Roboter beschäftigt sind. Falls der Roboter einen Lieferauftrag ausführt oder an ein bestimmtes Ziel fährt, sollen auch der angegebenene Ausgabe- und Lieferpunkt beziehungsweise das Ziel angezeigt werden. Weitere Daten, die von den Robotern stammen, sind die Positionen aller möglichen Ausgabe- und Lieferpunkten, sowie Ladestationen. Auch haben die Roboter festgelegte Pfade, an die Roboter sich beim Fahren orientieren. Diese Pfade sollen auch angezeigt werden können.

\subsubsection{Steuerung}

In der Steuerung sollen die Roboter beauftragt werden können. Da sich die Roboter von Pudu, die in dieser Arbeit eingesetzt werden, nicht autonom beladen können \cite{KettyBot2024}, braucht es hierfür die Angabe von Ausgabe- und Lieferpunkten. Am Ausgabepunkt wartet der Roboter dann bis er beladen wurde und fährt zum Lieferpunkt wo dann abgeladen wird. Außerdem soll es die Möglichkeit geben den Arbeitsmodus zu ändern, sodass der Roboter beispielsweise automatisch zufällige Stationen anfährt. Auch sollen die Roboter an ihre Ladestation geschickt werden können.

\subsubsection{Verwaltung}

In der Verwaltung sollen sowohl die Roboter als auch die Übersicht selbst verwaltet werden können. Das Ändern der Verschiedenen Roboter-Einstellungen soll möglich sein. Unter diese Roboter-Einstellungen fallen beispielsweise das Ändern der Ladestation und des Ausgabepunkts.

Es muss außerdem die Möglichkeit geben die 3D-Gebäudemodelle zu importieren und diese mit den Roboterdaten zu synchronisieren. Das Problem hierbei ist, dass hier praktisch zwei Gebäudekarten miteinander synchronisiert werden müssen: Das importierte 3D-Gebäudemodell und die mit \ac{VSLAM} erzeugte interne Karte der Roboter. Diese müssen aneinander angeglichen werden, damit die Position der Roboter in der Übersicht mit der Realität übereinstimmen.
% Aufteilung in Stockwerke erwähnen

\subsubsection{Benutzer}

Für die Anwendung gibt es drei verschiedene Nutzergruppen: Gäste, Kellner und Administratoren. Der Funktionsumfang erhöht sich in der angegebenen Reihenfolge. So sollen Gäste nur einen eingeschränkten Zugriff auf die Übersicht haben während Administratoren Zugriff auf alle Funktionen der Übersicht, Steuerung und Verwaltung haben.

Der Wert von Service Robotern muss aktuell insbesondere aus der Sicht von Gäste noch bewiesen werden \cite[S.~429]{Paluch2020}. Deshalb sollen Gäste einen eingeschränkten Zugriff auf die Übersicht der Roboter bekommen, in der sie die Positionen und aktuellen Aufträge sehen können. Mithilfe dieser Transparenz sollen den Kunden die Vorteile von Servicerobotern veranschaulicht werden.

Die Kellner sollen einen vollständigen Zugriff auf die Übersicht und Steuerung bekommen. So werden ihnen alle Funktionen zur Verfügung gestellt, die sie für den täglichen Betrieb brauchen. Auch sollen ihnen in der Übersicht mehr Informationen angezeigt werden. Einen Zugriff auf die Verwaltung der Roboter brauchen die Kellner nicht.

Auf die Verwaltung sollen nur Administratoren Zugriff haben. Während die Verwaltung der Roboter kontinuierlich gebraucht wird, wird die Verwaltung der Stockwerke hauptsächlich bei der Einrichtung gebraucht. So sollte die Karte nur zu Beginn eingerichtet werden, und danach nie wieder verändert werden müssen.

\subsubsection{Offlinefähigkeit}

Es ergibt keinen Sinn eine Offlinefähigkeit vorauszusetzen. Grund hierfür ist das die relevanten Funktionen eine Internetverbindung voraussetzen. Für alle zentralen Funktionen braucht es eine Verbindung zum \ac{BCB} über das Internet. Für den Überblick über die Roboter werden ihre Positionen dargestellt, die gecacht werden könnten. Da es sich aber um Echtzeitinformationen handelt, sind solche Daten schnell nicht mehr aktuell und somit nutzlos. Das kurzzeitige Cachen der Echtzeitinformationen kann aber trotzdem sinnvoll sein, um initiale Ladezeiten zu reduzieren. So können bereits Daten angezeigt werden, während die aktuellsten Daten noch geladen werden.

\subsection{Nicht-funktionale Anforderungen}

Lange Ladezeiten können insbesondere zur Hauptgeschäftszeit den Stress der Kellner erhöhen und eine große Quelle von Frust sein. Deshalb ist die Reduzierung der Ladezeiten eine zentrale Anforderung an den Prototyp. Die Webanwendung soll Inhalte schnell laden und dem Nutzer darstellen. Normalerweise wird hierfür sowohl die Performance auf Server- und Client-Seite verbessert. Da die zu entwickelnde Webanwendung aber hauptsächlich auf das bereits existierende \ac{BCB} zurückgegriffen wird, werden sich Optimierungen auf das Frontend beschränken. Hierfür eignen sich insbesondere Caching und andere Client-seitige Optimierungen wie das Verzögern des Ladens nicht essenzieller Ressourcen. Auch die Reduzierung des Datenverbrauchs ist hier relevant. Eine Reduzierung des Datenverbrauchs bietet auch den Vorteil, dass sich Kosten für Nutzer reduzieren, die einen teuren oder begrenzten Datenplan haben.

Unabhängig von Ladezeiten sollte die Anwendung trotz der Darstellung von 3D-Modellen sollte der Prototyp auf der Clientseite performant sein. So soll es möglichst keine Ruckler geben. Hierdurch wird auch die Benutzerfreundlichkeit verbessert. Die Benutzerfreundlichkeit soll auch gut sein.
