\newpage
\section{Anforderungen an den Prototyp}
In diesem Kapitel werden die Anforderungen 

\subsection{Abgrenzungen und Einschränkungen}

Ein Schwerpunkt dieser Arbeit liegt darin eine Methode zu wählen mit denen 3D-Modelle für den Prototyp generiert werden können. Die gewählte Methode fällt nicht in den Rahmen des Prototyps und kann von potentiellen Nutzern durch beliebige andere Methoden ausgetauscht werden, die kompatible 3D-Modelle generieren.

\subsection{Funktionale Anforderungen}

Wie bereits in der Einleitung beschrieben soll der Prototyp als Webanwendung implementiert werden. So wird eine Plattformunabhängigkeit garantiert. Diese wird gebraucht, da die Anwendung von Kellnern auf beliebigen Smartphones genutzt werden soll. Auch an Desktop Computern kann die Anwendung insbesondere zur Verwaltung der Roboter genutzt werden können. Entsprechend soll die Webanwendung Responsive sein um auf Gerätetypen mit verschiedenen Bildschirmgrößen zu funktionieren. 



Es ergibt keinen Sinn die Offlinefähigkeit vorauszusetzen. Grund hierfür ist das die relevanten Funktionen eine Internetverbindung voraussetzen. Für die Steuerung und Verwaltung der Roboter braucht es eine Verbindung zum \ac{BCB} über das Internet. Für den Überblick über die Roboter werden Echtzeitinformationen dargestellt, die gecached werden könnten. Da es sich aber um Echtzeitinformationen handelt sind solche Daten schnell nicht mehr aktuell und somit nutzlos. Das kurzzeitige Cachen der Echtzeitinformationen kann aber trotzdem sinnvoll sein um initiale Ladezeiten zu reduzieren. So können bereits Daten angezeigt werden, während die aktuellen Daten noch geladen werden.

\subsection{Nicht-funktionale Anforderungen}

Lange Ladezeiten können insbesondere zur Hauptgeschäftszeit den Stress der Kellner erhöhen und eine große Quelle von Frust sein. Deshalb ist die Reduzierung der Ladezeiten eine Anforderung an den Prototyp. Die Webanwendung soll Inhalte schnell laden und dem Nutzer darstellen. Normalerweise wird hierfür sowohl die Performance auf Server- und Client-Seite verbessert. Da die zu entwickelnde Webanwendung aber hauptsächlich auf das bereits existierende \ac{BCB} zurückgegriffen wird, werden sich Optimierungen auf das Frontend beschränken. Hierfür eignen sich insbesondere Caching und andere Client-seitige Optimierungen wie das Verzögern des Ladens nicht essentieller Ressourcen. Auch die Reduzierung des Datenverbrauchs ist hier relevant. Dies bietet auch den Vorteil, dass sich Kosten für Nutzer reduzieren, die einen teuren oder begrenzten Datenplan haben.

Unabhängig von Ladezeiten sollte die Anwendung 

\subsection{Qualitätskriterien}
