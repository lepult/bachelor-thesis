\newpage
\section{Anforderungen an den Prototyp}
In diesem Kapitel werden die Anforderungen an den Prototyp vorgestellt.

\subsection{Anforderungen an Modellerzeugung}

Für die Relevanz des Prototyps ist es erheblich, dass es eine einfache Methode gibt, mit der 3D-Gebäudemodelle erzeugt werden können. Hierfür soll eine Methode gefunden werden, die diese Anforderung erfüllt. Eine tiefere Analyse der gewählten Methode sowie ein tieferer Vergleich zwischen verschiedenen Methoden ist unerheblich, da die gewählte Methode nicht die beste sein muss. In dieser Arbeit soll nur herausgearbeitet werden, dass es eine passende Methode gibt, nicht welche am besten für den Anwendungsfall geeignet ist. Die Methode soll möglichst wenig kosten und einen geringen Aufwand mit sich bringen. Auch sollte die Methode kein technisches Know-How erfordern. Die erzeugten Modelle müssen nicht hundertprozentig genau, aber genau genug sein, um eine gute Übersicht zu ermöglichen.

\subsection{Funktionale Anforderungen}\label{sec:FunctionalRequirements}
Wie bereits in der Einleitung beschrieben, soll der Prototyp als Webanwendung implementiert werden. So wird eine Plattformunabhängigkeit mit geringem Aufwand garantiert. Diese wird gebraucht, damit die Anwendung auf beliebigen Geräten genutzt werden kann. Die Anwendung soll sowohl auf Smartphones als auch auf Desktop Computern genutzt werden können. Entsprechend soll der Prototyp responsiv sein, um auf Gerätetypen mit beliebigen Bildschirmgrößen zu funktionieren.

Die Anwendung lässt sich in drei zentrale Funktionen aufteilen: Übersicht, Steuerung und Verwaltung.

\subsubsection{Übersicht}

In der Übersicht sollen die relevanten Roboterdaten in einer dreidimensionalen Ansicht zusammen mit dem Gebäudemodell abgebildet werden. Da die Roboter in der Lage sind Fahrstuhl zu fahren soll es die Möglichkeit geben zwischen verschiedenen Stockwerken zu navigieren. Es soll außerdem in Echtzeit angezeigt werden, wo sich die Roboter befinden. In der Darstellung sollen die Roboter dann so wie in der Realität fahren. Auch soll ersichtlich sein, ob und womit die Roboter beschäftigt sind. Falls der Roboter einen Lieferauftrag ausführt soll das Ziel angezeigt werden. Weitere Daten, die von den Robotern stammen und angezeigt werden sollen, sind die Positionen aller möglichen Ausgabe- und Lieferpunkten, sowie Ladestationen. Auch haben die Roboter festgelegte Pfade, an denen sie sich beim Fahren orientieren. Diese Pfade sollen auch angezeigt werden.

\subsubsection{Steuerung}

In der Steuerung sollen die Roboter beauftragt werden können. So soll es die Möglichkeit geben einen Zielpunkt einzustellen. Auch sollen die Roboter an ihre Ladestation geschickt werden können.

\subsubsection{Verwaltung}

In der Verwaltung sollen sowohl die Roboter als auch die Übersicht selbst verwaltet werden können. Das Ändern der Verschiedenen Roboter-Einstellungen soll möglich sein. So soll beispielsweise die eingestellte Ladestation geändert werden können.

Es muss außerdem die Möglichkeit geben die 3D-Gebäudemodelle zu importieren und diese mit den Roboterdaten zu synchronisieren. Das Problem hierbei ist, dass zwei Gebäudekarten miteinander synchronisiert werden müssen: Das importierte 3D-Gebäudemodell und die mit \ac{VSLAM} erzeugte interne Karte der Roboter. Diese müssen aneinander angeglichen werden, damit die Positionen der Roboter in der Übersicht mit der Realität übereinstimmen.
% Aufteilung in Stockwerke erwähnen

\subsubsection{Benutzer}

Für die Anwendung gibt es drei verschiedene Nutzergruppen: Gäste, Mitarbeiter und Administratoren. Der Funktionsumfang erhöht sich in der angegebenen Reihenfolge. So sollen Gäste nur einen eingeschränkten Zugriff auf die Übersicht haben während Administratoren Zugriff auf alle Funktionen der Übersicht, Steuerung und Verwaltung haben.

Der Wert von Service Robotern muss aktuell insbesondere aus der Sicht von Gästen noch bewiesen werden \cite[S.~429]{Paluch2020}. Deshalb sollen Gäste einen eingeschränkten Zugriff auf die Übersicht der Roboter bekommen, in der sie die Positionen und aktuellen Aufträge sehen können. Mithilfe dieser Transparenz sollen den Kunden die Vorteile von Servicerobotern veranschaulicht werden.

Die Mitarbeiter sollen einen vollständigen Zugriff auf die Übersicht und Steuerung bekommen. So werden ihnen alle Funktionen zur Verfügung gestellt, die sie für das Steuern der Roboter brauchen. Auch sollen ihnen in der Übersicht mehr Informationen angezeigt werden. Einen Zugriff auf die Verwaltung der Roboter brauchen die Mitarbeiter nicht.

Auf die Verwaltung sollen nur Administratoren Zugriff haben. Während die Verwaltung der Roboter kontinuierlich genutzt wird, wird die Verwaltung der Stockwerke hauptsächlich bei der Einrichtung gebraucht. So sollte die Karte nur zu Beginn eingerichtet und danach nie wieder verändert werden müssen.

\subsection{Nicht-funktionale Anforderungen}
Lange Ladezeiten können bei der Nutzung der Anwendung eine große Quelle von Frust sein. Deshalb ist die Reduzierung der Ladezeiten eine zentrale Anforderung an den Prototyp. Die Webanwendung soll Inhalte schnell laden und dem Nutzer darstellen. Normalerweise wird die Performance hiefür sowohl auf der Server- als auch auf der Client-Seite verbessert. Da der zu entwickelnde Prototyp aber hauptsächlich auf das bereits existierende \ac{BCB} zugreifen wird, werden sich Optimierungen auf das Frontend beschränken müssen. Hierfür eignen sich Client-seitige Optimierungen wie das Verzögern des Ladens nicht essenzieller Ressourcen, sowie die Reduzierung des Datenverbrauchs. Eine Reduzierung des Datenverbrauchs bietet auch den Vorteil, dass sich Kosten für Nutzer reduzieren, die einen teuren oder begrenzten Datenplan haben. Unabhängig von Ladezeiten sollte der Prototyp trotz der aufwändigen Darstellung von 3D-Modellen möglichst performant sein. So soll es beispielsweise beim Navigieren keine Ruckler geben. Hierdurch wird auch die Benutzerfreundlichkeit verbessert. Diese Anforderungen lassen sich unter dem Begriff der Performance zusammenfassen. Neben einer guten Performance soll es auch eine gute Usability geben, damit sich der Nutzer gut zurechtfinden kann und bei der Nutzung kein Frust entsteht.
